\documentclass[12pt]{article}



\usepackage{t1enc}
\usepackage[utf8]{inputenc}
\usepackage[magyar]{babel}
\usepackage{amsmath}
\usepackage{amsfonts}
\usepackage{amsthm}
\usepackage{amssymb}
\usepackage{enumerate}
\usepackage[shortlabels]{enumitem}
\usepackage[autostyle]{csquotes}
\usepackage{graphicx}



\selectlanguage{magyar}

\DeclareQuoteAlias{dutch}{magyar}

\theoremstyle{definition}
\newtheorem{definition}{Definíció}
\newtheorem*{definition*}{Definíció}

\theoremstyle{plain}
\newtheorem{theorem}{Tétel}
\newtheorem*{theorem*}{Tétel}

\let\oldemptyset\emptyset
\let\emptyset\varnothing

\newcommand{\forwardhat}{\overset{\rightharpoonup}}
\newcommand{\backwardhat}{\overset{\leftharpoonup}}

\newcommand{\inc}{\textit{inc}}
\newcommand{\dec}{\textit{dec}}

\graphicspath{ {./images/} }



\title{Reverzibilis Reaction System}
\date{\today}

\begin{document}
    \maketitle

    \begin{definition*}
        Legyen $\mathcal{A} = (S, A)$ egy \textit{reaction system} és $\pi$ egy $\mathcal{A}$ felett adott $\pi$ \textit{interactive process}, amelynek állapotait $\textit{sts}(\pi)=W_{0},W_{1},\ldots W_{n}$ módon jelöljük. $\pi$ reverzibilis, amennyiben minden $W_{i}$ ($1 \leq i \leq n$) állapotára teljesül, hogy $\nexists \; W \in \mathcal{P}(S) : res_{\mathcal{A}}(W) = W_{i} \; \wedge W \neq W_{i - 1}$.
    \end{definition*}

    \begin{definition*}
        Legyen $\mathcal{A} = (S, A)$ egy \textit{reaction system}. $\mathcal{A}$ reverzibilis, ha bármely felette definiált $\pi$ \textit{interactive process} reverzibilis.
    \end{definition*}

    \begin{theorem*}
        Az $\mathcal{A} = (S, A)$ \textit{reaction system} reverzibilis, amennyiben teljesülnek a következő feltételek:
        \begin{enumerate}[label={(\arabic*)}]
            \item
            Egyértelmű, hogy egy állapot mely reakciók alkalmazásával állt elő. Azaz, tetszőleges $a = (R_{a}, I_{a}, P_{a}), b = (R_{b}, I_{b}, P_{b}) \in A$ ($a \neq b$) reakciópár esetén a következők egyike teljesül:
            \begin{itemize}
                \item
                $a$ és $b$ azonos feltételek mellett alkalmazhatók, tehát $R_{a} = R_{b}$ és $I_{a} = I_{b}$.

                \item
                $a$ és $b$ produktumai nem átfedők, azaz $P_{a} \cap P_{b} = \emptyset$.

                \item
                $a$ produktuma tartalmazza $b$ produktumát is, azonban $a$ és $b$ nem alkalmazhatók egyszerre, tehát $P_{b} \subset P_{a}$ és $R_{a} \cap I_{b} \neq \emptyset$ vagy $R_{b} \cap I_{a} \neq \emptyset$.

                \item
                $a$ és $b$ produktumai megegyezők, azonban van olyan $c = (R_{c}, I_{c}, P_{c}) \in A$ szabály, mely $a$-val együtt mindig, $b$-vel együtt azonban sosem alkalmazható. Ekkor $R_{c} \subseteq R_{a}$ és $I_{c} = I_{a}$, továbbá $R_{c} \cap I_{b} \neq \emptyset$.
            \end{itemize} 

            \item
            A kontextusból kapott szimbólumok nem állhatnak elő egy reakció produktumaként sem: ha $\pi = (\gamma, \delta)$ egy \textit{interactive process}, ahol $\gamma = C_{0}, C_{1}, \ldots, C_{n}$, $n \geq 1$, akkor bármely $C_{i}$ kontextus és $a \in A$ reakció esetén $C_{i} \cap P_{a} = \emptyset$.

            \item
            Az állapotok minden eleme részt vesz valamilyen reakcióban: ha $\pi$ egy \textit{interactive process}, ahol $\textit{sts}(\pi) = W_{0}, W_{1}, \ldots, W_{n}$, $n \geq 1$, akkor $\bigcup_{a \in \textit{en}(W_{i})} R_{a} = W_{i}$, $i \leq n$.
        \end{enumerate}
    \end{theorem*}

    \begin{definition*}
        A reverzibilitás szimulálása.
    \end{definition*}
\end{document}
