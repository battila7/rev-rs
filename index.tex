\documentclass[12pt]{article}



\usepackage{t1enc}
\usepackage[utf8]{inputenc}
\usepackage[magyar]{babel}
\usepackage{amsmath}
\usepackage{amsfonts}
\usepackage{amsthm}
\usepackage{amssymb}
\usepackage{enumerate}
\usepackage{mathtools}
\usepackage[shortlabels]{enumitem}
\usepackage[autostyle]{csquotes}
\usepackage{graphicx}
\usepackage{mathrsfs}



\selectlanguage{magyar}

\DeclareQuoteAlias{dutch}{magyar}

\theoremstyle{definition}
\newtheorem{definition}{Definíció}
\newtheorem*{definition*}{Definíció}

\theoremstyle{remark}
\newtheorem{remark}{Megjegyzés}
\newtheorem*{remark*}{Megjegyzés}

\theoremstyle{plain}
\newtheorem{theorem}{Tétel}
\newtheorem*{theorem*}{Tétel}

\theoremstyle{plain}
\newtheorem{lemma}{Lemma}
\newtheorem*{lemma*}{Lemma}

\let\oldemptyset\emptyset
\let\emptyset\varnothing

\newcommand{\forwardhat}{\overset{\rightharpoonup}}
\newcommand{\backwardhat}{\overset{\leftharpoonup}}

\newcommand{\inc}{\textit{inc}}
\newcommand{\dec}{\textit{dec}}

\newcommand{\en}{\textit{en}}
\newcommand{\res}{\textit{res}}
\newcommand{\ip}{\textit{ip}}
\newcommand{\excl}{\; \textit{excl} \;}
\newcommand{\incl}{\; \textit{incl} \;}

\newcommand{\reaction}[3]{
    (#1, \, #2, \, #3)
}

\graphicspath{ {./images/} }



\title{Reverzibilis Reaction System}
\date{\today}

\begin{document}
    \maketitle

    \section*{Előzetes ismeretek}

    \begin{definition*}
        Legyen $\mathscr{A} = (S, A)$ egy \textit{reaction system}. Egy $\mathscr{A}$-beli \textit{interactive process} véges sorozatok olyan $\pi = (\gamma, \delta)$ párja, amelyben $\gamma = C_{0}, C_{1}, \ldots C_{n}$, $\delta = D_{1}, \ldots D_{n}$, $n \geq 1$, ahol $C_{0}, \ldots C_{n}$, $D_{1}, \ldots D_{n} \subseteq S$, $D_{1} = \textit{res}_{\mathscr{A}}(C_{0})$ és $D_{i} = \textit{res}_{\mathscr{A}}(D_{i - 1} \cup C_{i - 1})$, ha $2 \leq i \leq n$.
    \end{definition*}

    \begin{definition*}
        Legyen $a = \reaction{R_{a}}{I_{a}}{P_{a}}$ egy reakció, $T$ pedig egy véges halmaz. Azt mondjuk, hogy $a$ alkalmazható $T$-re, ha $R_{a} \subseteq T$ és $I_{a} \cap T = \emptyset$. Erre a továbbiakban az $\en_{a}(T)$ jelölést használjuk. Az $a$ reakció által a $T$ halmazból képzett eredményhalmazt $\res_{a}(T)$ módon jelöljük, és a következőképpen definiáljuk: $\res_{a}(T) = P_{a}$, ha $\en_{a}(T)$, egyébként pedig $\res_{a}(T) = \emptyset$.
    \end{definition*}

    \begin{definition*}
        Legyen $A$ reakciók egy véges halmaza és $T$ egy véges halmaz. Ekkor az $A$ reakcióhalmaz $T$-re való alkalmazásának eredménye $\res_{A}(T) = \bigcup_{a \,\in\, A}\res_{a}(T)$.
    \end{definition*}

    \begin{definition*}
        Legyen $A$ reakciók egy véges halmaza és $T$ egy véges halmaz. Jelölje ekkor $\textit{RES}_{A}(T)$ azt a halmazt, mely az összes, $T$-ből $A$-beli reakciók alkalmazásával előállítható eredményhalmazt tartalmazza, azaz $\textit{RES}_{A}(T) = \{ \, \res_{A}(W) \; \mid \; W \subseteq T\, \}$
    \end{definition*}

    \begin{definition*}
        Legyen $A$ reakciók egy véges halmaza és $T$ egy véges halmaz. $A$ azon részhalmazát, mely csak olyan reakciókat tartalmaz, melyek alkalmazhatók $T$-re, $\en_{A}(T)$ módon jelöljük.
    \end{definition*}

    \begin{definition*}
        Legyen $S$ egy véges halmaz, $A$ pedig reakciók egy véges halmaza. Jelölje ekkor $\textit{EN}_{A}(S)$ azon különböző reakcióhalmazokat, melyek alkalmazhatók $S$ valamely részhalmazára, azaz
        \begin{align*}
            \textit{EN}_{A}(S) = \{ B \;|\; B \subseteq A \wedge \exists\,T \subseteq S : \en_{B}(T) = B \}
        \end{align*}
    \end{definition*}
    
    \begin{remark*}
        A következőkben, az egyszerűbb olvashatóság érdekében, feltesszük, hogy ha adott egy $a \in A$ reakció, akkor annak komponenseit $a$ alsó indexbe helyezésével helyezésével jelöljük, az $a = \reaction{R_{a}}{I_{a}}{P_{a}} \in A$ kifejezés kiírása nélkül.
    \end{remark*}

    \begin{remark*}
        A következőkben feltesszük, hogy egy $\mathscr{A} = (S, A)$ \textit{reaction system} $S$ elemhalmazát (\textit{background set}) megadhatjuk két (nem feltétlenül diszjunkt) halmaz uniójaként: $S = \Sigma_{p} \cup \Sigma_{c}$. Jelölje ekkor $\Sigma_{p}$ azon elemek halmazát, melyek előállhatnak valamely $A$-beli reakció alkalmazásával, azaz $\Sigma_{p} = \bigcup_{a \, \in \, A}P_{a}$. Jelölje továbbá $\Sigma_{c}$ azon elemek halmazát, melyek szerepelhetnek valamilyen $A$-beli \textit{interactive process} valamely bemeneti halmazában. Utóbbi azt jelenti, hogy korlátozható azon elemek halmaza, melyek bemenetként megjelenhetnek.
    \end{remark*}

    \section*{Reverzibilis Reaction System}

    \begin{remark*}
        A továbbiakban, az általánosság elvesztése nélkül, kizárólag olyan $\mathscr{A} = (S, A)$ \textit{reaction systemeket} fogunk tekinteni, melyek nem tartalmaznak azonos feltételek mellett alkalmazható reakciókat. Ez azt jelenti, hogy nincs két olyan $a, b\in A$ ($a \neq b$) reakció, melyekre teljesülne, hogy $R_{a} = R_{b}$ és $I_{a} = I_{b}$. Az ilyen $a, b$ reakciókat ugyanis egyszerűen összevonhatjuk egy $c$ reakcióvá, ahol $c = (R_{a}, I_{a}, P_{a} \cup P_{b})$.
    \end{remark*}

    \begin{definition*}
        Legyen $\mathscr{A} = (S, A)$ egy \textit{reaction system}, $\pi=(\gamma, \delta)$ pedig egy \textit{interactive process} $\mathscr{A}$-ban a $\gamma = C_{0}, C_{1}, \ldots C_{n}$, bemeneti halmazokkal és $\textit{sts}(\pi)=W_{0},W_{1},\ldots W_{n}$ állapothalmazokkal. $\pi$ reverzibilis, amennyiben minden $W_{i}$ ($1 \leq i \leq n$) állapotára teljesül, hogy ha $W \subseteq S$ olyan, hogy $res_{\mathscr{A}}(W) \cup C_{i} = W_{i}$, akkor $W = W_{i - 1}$.
    \end{definition*}

    \begin{lemma*}
        Legyen $\mathscr{A} = (S, A)$ egy \textit{reaction system}, $\pi$ pedig egy \textit{interactive process} $\mathscr{A}$-ban az $\textit{sts}(\pi)=W_{0},W_{1},\ldots W_{n}$ állapothalmazokkal. $\pi$ csak akkor lehet reverzibilis, ha az állapothalmazainak minden eleme részt vesz valamilyen reakcióban, azaz $\bigcup_{a \in \textit{en}(W_{i})} R_{a} = W_{i}$, $i \leq n$.
    \end{lemma*}

    \begin{proof}
        Legyen $\mathscr{A} = (S, A)$ egy \textit{reaction system} és $\pi=(\gamma, \delta)$ egy \textit{interactive process} $\mathscr{A}$-ban a $\gamma = C_{0}, C_{1}, \ldots C_{n}$, bemeneti halmazokkal és az $\textit{sts}(\pi)=W_{0},W_{1},\ldots W_{n}$ állapothalmazokkal. Legyen $W_{i - 1} \subseteq S$ és $W_{i} \subseteq S$ $\pi$ két egymást követő állapothalmaza, azaz $\res_{A}(W_{i-1}) \cup C_{i}=W_{i}$.
        
        Tegyük fel, hogy van olyan elem a $W_{i-1}$ halmazban, mely nem vesz részt egy reakcióban sem:
        \begin{align*}
            \bigcup_{a \,\in\, \textit{en}(W_{i-i})} R_{a} \subset W_{i-1}.
        \end{align*}
        Ez pontosan azt jelenti, hogy létezik egy olyan $W \subset S$ halmaz, mely
        \begin{align*}
            W = \bigcup_{a \,\in\, \textit{en}(W_{i-i})} R_{a}
        \end{align*}
        módon adott, és melyre teljesül, hogy
        \begin{align*}
            \res_{A}(W) \cup C_{i} = W_{i}.
        \end{align*}

        Ekkor $\pi$ biztosan nem lehet reverzibilis, hiszen $W \neq W_{i - 1}$, viszont $W_{i} = \res_{A}(W) \cup C_{i} = \res_{A}(W_{i - 1}) \cup C_{i}$.
    \end{proof}

    \begin{definition*}
        Egy $\mathscr{A}$ \textit{reaction system} reverzibilis, amennyiben minden $\mathscr{A}$-beli $\pi$ \textit{interactive process} reverzibilis.
    \end{definition*}

    \begin{lemma*}
        Legyen $\mathscr{A} = (S, A)$ egy \textit{reaction system}, $\mathcal{D} \subseteq \mathcal{P}(S)$ pedig a következő módon adott:
        \begin{align*}
            \mathcal{D} = \{ \, R \setminus \Sigma_{c} \mid R \in \textit{RES}_{A}(S) \, \}.
        \end{align*}
        $\mathscr{A}$ csak akkor lehet reverzibilis, ha
        \begin{align*}
            |\,\mathcal{D}\,| = |\,\textit{RES}_{A}(S)\,|.
        \end{align*}
    \end{lemma*}

    \begin{proof}
        Legyen $\mathscr{A} = (S, A)$ egy \textit{reaction system}. Tegyük fel, hogy 
        \begin{align*}
            |\,\mathcal{D}\,| \neq |\,\textit{RES}_{A}(S)\,|.
        \end{align*}
        Ekkor biztos, hogy
        \begin{align*}
            |\,\mathcal{D}\,| < |\,\textit{RES}_{A}(S)\,|.
        \end{align*}
        Ez azt jelenti, hogy létezik $R_{i}, R_{j} \in \textit{RES}_{A}(S)$ ($R_{i} \neq R_{j}$), hogy
        \begin{align*}
            R_{i} \setminus \Sigma_{c} = R_{j} \setminus \Sigma_{c}.
        \end{align*}
        Mivel $R_{i}$ és $R_{j}$ elemei a $\textit{RES}_{A}(S)$ halmaznak, ezért léteznek olyan $W_{i}, W_{j} \subseteq S$ ($W_{i} \neq W_{j}$) halmazok, hogy $\res_{A}(W_{i}) = R_{i}$ és $\res_{A}(W_{j}) = R_{j}$. Továbbá, mivel $R_{i} \neq R_{j}$, azonban $R_{i} \setminus \Sigma_{c} = R_{j} \setminus \Sigma_{c}$, ezért $C$-t olyannak választva, hogy $C = (R_{i} \cap \Sigma_{c}) \cup (R_{j} \cap \Sigma_{c})$, teljesül, hogy
        \begin{align*}
            R_{i} \cup C = R_{j} \cup C.
        \end{align*}
        Ezen állítások azt jelentik, hogy létezik $W_{i}, W_{j} \subseteq S$ ($W_{i} \neq W_{j}$) és $C \subseteq \Sigma_{c}$ úgy, hogy
        \begin{align*}
            \res_{A}(W_{i}) \cup C = \res_{A}(W_{j}) \cup C,
        \end{align*}
        azaz $\mathscr{A}$ nem reverzibilis.
    \end{proof}     

    \begin{lemma*}
        Legyen $\mathscr{A} = (S, A)$ egy \textit{reaction system}. $\mathscr{A}$ csak akkor lehet reverzibilis, ha egyértelmű, hogy egy eredményhalmaz mely reakciók alkalmazásával állt elő. Azaz bármely $E_{i}, E_{j} \in \textit{EN}_{A}(S)$ ($E_{i} \neq E_{j}$) esetén
        \begin{align*}
            \bigcup\limits_{a \,\in\, E_{i}} P_{a} \neq \bigcup\limits_{b \,\in\, E_{j}} P_{b}.
        \end{align*}
    \end{lemma*}

    \begin{proof}
        Legyen $\mathscr{A} = (S, A)$ egy \textit{reaction system}. Tegyük fel, hogy létezik $E_{i}, E_{j} \in \textit{EN}_{A}(S)$ ($E_{i} \neq E_{j}$) úgy, hogy
        \begin{align*}
            \bigcup\limits_{a \,\in\, E_{i}} P_{a} = \bigcup\limits_{b \,\in\, E_{j}} P_{b}.
        \end{align*}
        Ha $E_{i}, E_{j} \in \textit{EN}_{A}(S)$ és $E_{i} \neq E_{j}$, akkor biztosan léteznek olyan $W_{i}, W_{j} \subseteq S$ halmazok, hogy $\en_{A}(W_{i}) = E_{i}$, $\en_{A}(W_{j}) = E_{j}$ és $W_{i} \neq W_{j}$. Ugyanakkor, mivel
        \begin{align*}
            \bigcup\limits_{a \,\in\, E_{i}} P_{a} = \bigcup\limits_{b \,\in\, E_{j}} P_{b},
        \end{align*}
        ezért ez pontosan azt jelenti, hogy létezik $W_{i}, W_{j} \subseteq S$ ($W_{i} \neq W_{j}$) úgy, hogy
        \begin{align*}
            \res_{A}(W_{i}) = \res_{A}(W_{j}),
        \end{align*}
        azaz $\mathscr{A}$ nem reverzibilis.
    \end{proof}

    \begin{theorem*}
        Az $\mathscr{A} = (S, A)$ \textit{reaction system} reverzibilis, amennyiben a következő feltételek mindegyikének eleget tesz:
        \begin{enumerate}[label={(\arabic*)}]
            \item
            Az $A$ reakcióhalmaz elemei olyanok, hogy bármely $\pi$ \textit{interactive process} teljesíti az elemek eltűnésére vonatkozó lemmát.

            \item
            Az $S$ halmaz két diszjunkt részhalmazból tevődik össze:
            \begin{align*}
                S = \Sigma_{p} \cup \Sigma_{c} \qquad \Sigma_{p} \cap \Sigma_{c} = \emptyset,
            \end{align*}
            mely halmazokra teljesül, hogy $\Sigma_{c} \cap \bigcup_{a \in A} P_{a} = \emptyset$ és $\Sigma_{p} \cap \bigcup_{0 \leq i \leq n} C_{i} = \emptyset$ bármely $\mathscr{A}$-beli $\pi$ \textit{interactive processre}.

            \item
            Az $A$ reakcióhalmaz elemei olyanok, hogy a megfelelő lemma szerint egyértelmű, hogy egy eredményhalmaz mely reakciók alkalmazásával állt elő.
        \end{enumerate}
    \end{theorem*}

    \begin{proof}
        Indirekt módon tegyük fel, hogy adott egy olyan $\mathscr{A}=(S, A)$ ($S = \Sigma_{p} \cup \Sigma_{c}$) \textit{reaction system}, mely teljesíti a fenti tételt, azonban nem reverzibilis. Ekkor létezik olyan $\pi$ \textit{interactive process} $\mathscr{A}$-ban, mely nem reverzibilis. Ez azt jelenti, hogy a folyamat tartalmaz olyan $W \subseteq S$ állapotot, melyhez léteznek olyan $W_{i} \subseteq S$ és $W_{j} \subseteq S$ ($W_{i} \neq W_{j}$) halmazok, hogy
        \begin{align*}
            \res_{A}(W_{i}) \cup C &= W \\
            \res_{A}(W_{j}) \cup C &= W,
        \end{align*}
        ahol $C \subseteq S_{c}$ a $W$ állapothoz tartozó bemeneti halmaz. 
        
        Mivel mind $\res_{A}(W_{i}) \subseteq S_{d}$, mind $\res_{A}(W_{j}) \subseteq S_{d}$, ezért
        \begin{align*}
            \res_{A}(W_{i}) &= W \setminus C \\
            \res_{A}(W_{j}) &= W \setminus C, \\
        \end{align*}
        azaz
        \begin{align*}
            \res_{A}(W_{i}) = \res_{A}(W_{j}).
        \end{align*}

        Ez azt jelenti, hogy
        \begin{align*}
            \bigcup\limits_{a \,\in\, \en_{A}(W_{i})}P_{a} = \bigcup\limits_{a \,\in\, \en_{A}(W_{j})}P_{a}.
        \end{align*}

        $\mathscr{A}$ teljesíti a fenti tételt,  ezért ilyen esetben
        \begin{align*}
            \en_{A}(W_{i}) = \en_{A}(W_{j}).
        \end{align*}

        Egy számítási lépésben egy adott állapothalmaz minden elemének részt kell vennie legalább egy reakcióban, azaz
        \begin{align*}
            \bigcup\limits_{a \,\in\, \en_{A}(W_{i})} R_{a} = W_{i}
        \end{align*}
        és
        \begin{align*}
            \bigcup\limits_{a \,\in\, \en_{A}(W_{j})} R_{a} = W_{j}.
        \end{align*}

        A bizonyítás elején feltettük, hogy $W_{i} \neq W_{j}$, amiből következően
        \begin{align*}
            \bigcup\limits_{a \,\in\, \en_{A}(W_{i})} R_{a} \neq \bigcup\limits_{a \,\in\, \en_{A}(W_{j})} R_{a}.
        \end{align*}

        Tudjuk, hogy $\en_{A}(W_{i}) = \en_{A}(W_{j})$, mely halmazt jelöljük $E$-vel. Ezt a megelőző kifejezésbe beírva kapjuk, hogy
        \begin{align*}
            \bigcup\limits_{a \,\in\, E} R_{a} \neq \bigcup\limits_{a \,\in\, E} R_{a}.
        \end{align*}
        
        Ez azonban ellentmondás, azaz $\mathscr{A}$ reverzibilis lesz.
    \end{proof}

    \subsection*{Példák}

    \subsubsection*{Reverzibilis bináris számláló}

    Reverzibilis \textit{reaction system} felhasználásával megvalósítható egy olyan ciklikus bináris számláló, melynek értéke az előrefelé számítás során növekszik, míg a hátrafelé számítás során csökken. A ciklikusság azt jelenti, hogy a legnagyobb ábrázolható érték további növelése a $0$ értéket, míg a $0$ érték további csökkentése a legnagyobb ábrázolható értéket eredményezi.

    Először is tegyük fel, hogy adott egy $n > 0$ egész. $n$ jelenti a számláló bithosszát. Ekkor a \textit{reaction system} alaphalmaza a következő lesz:
    \begin{align*}
        S_{n} = \{ p_{0}, p_{1}, \ldots, p_{n - 1}\} \cup \{ \textit{inc}, z \}.
    \end{align*}
    A fenti halmaz $p_{i}$ elemei reprezentálják az egyes bitek beállított (azaz $1$ értékű) állapotát, míg az $\textit{inc}$ elemmel a számláló értékének növelését válthatjuk ki. A $z$ elem jelenti a számláló $0$ értékét.

    A számábrázolás tehát a következőképpen alakul. Tegyük fel, hogy a \textit{reaction system} egy $M \subseteq S$ állapotban van. Ekkor, ha $p_{i} \in M$, akkor az $i$-edik pozíción levő bit $1$ értékkel, amennyiben pedig $p_{i} \notin M$, akkor $0$ értékkel rendelkezik. Például, ha $n = 4$ és $M = \{p_{2}, p_{0}\}$, akkor a \textit{reaction system} állapota a $0101$ bináris számot írja le.
    
    Előrefelé számítás során az $\textit{inc}$ elemet használhatjuk a számláló értékének eggyel történő növelésére. Egyszerű példát tekintve, ez azt jelenti, hogy amennyiben a \textit{reaction system} egy $\{p_{1}, p_{0}, \textit{inc}\}$ állapotban van, akkor valamely reakciók végrehajtása után a $\{ p_{2} \}$ állapotba kell kerülnie.

    Folytassuk tehát az említett működéshez szükséges reakciók megadásával! Legyen $n > 0$ adott, ekkor a reakciók $A_{n}$ halmaza a következő elemekből áll:
    \begin{align*}
        a_{0} &= \reaction{\{ \inc, z \}}{O_{2^{n}-1}}{O_{1}} \\
        a_{i} &= \reaction{\{ \inc \} \cup O_{i}}{Z_{i}}{O_{i + 1}}, \qquad 1 \leq i < 2^{n} - 2, \\
        a_{2^{n}-1} &= \reaction{\{ \inc \} \cup O_{2^{n} - 1}}{\{z\}}{\{z\}}
    \end{align*}
    ahol
    \begin{align*}
        O_{i} &= \{ \; p_{j} \; : \; \text{a $j$-edik bit értéke $1$ $i$ bináris felbontásában} \; \}, \\
        Z_{i} &= S \setminus\{ \inc \} \setminus O_{i}.
    \end{align*}
    Az egyes reakciók megfelelnek a számláló értékének $i$-ről $i + 1$-re történő növelésének (kivétel az utolsó reakció, mely a számláló átfordulását eredményezi).

    Az $n$-bites számlálónak megfelelő \textit{reaction system} ekkor $\mathcal{B}_{n}=(S_{n}, A_{n})$.

    Tekintsünk most egy példát! Tegyük fel, hogy egy kétbites számlálót szeretnénk készíteni, azaz $n = 2$. Az $S_{2}$ alaphalmaz ekkor a $\{p_{1}, p_{0}, \inc, z \}$ elemekből áll, a reakciók $A_{2}$ halmazát pedig az
    \begin{align*}
        a_{0} &= \reaction{\{ \inc, z \}}{\{p_{0}, p_{1}\}}{\{ p_{0} \}}, \\
        a_{1} &= \reaction{\{ \inc, p_{0} \}}{\{ z, p_{1} \}}{\{ p_{1} \}}, \\
        a_{2} &= \reaction{\{ \inc, p_{1} \}}{\{ z, p_{0} \}}{\{ p_{1}, p_{0} \}}, \\
        a_{3} &= \reaction{\{ \inc, p_{0}, p_{1} \}}{\{z\}}{\{z\}}
    \end{align*}
    elemek alkotják.

    Ha az egymást követő kontextushalmazok sorra az \inc növelő elemből állnak, akkor a $\mathcal{B}_{2} = (S_{2}, A_{2})$ \textit{reaction system} a következő állapotokat fogja kiszámolni:
    \begin{align*}
        \{z\} \rightarrow \{p_{0}\} \rightarrow \{p_{1}\} \rightarrow \{p_{1}, p_{0}\} \rightarrow \{z\} \rightarrow \cdots
    \end{align*}


    \section*{A reverzibilitás szimulálása}

    \begin{definition*}
        Tegyük fel, hogy adott egy $n + 1$ halmazból álló véges $W=W_{0}, W_{1}, \ldots, W_{n}$ halmazsorozat. Ekkor a halmazsorozat $n + 1$ hosszát jelölje $|W|$.
    \end{definition*}

    \begin{definition*}
        Tegyük fel, hogy adottak a $V_{0}, V_{1}, \ldots, V_{n}$ ($n \geq 0$) véges halmazsorozatok. Ekkor az e halmazsorozatok összefűzésével kapott halmazsorozatot $W = V_{0}, V_{1}, \ldots, V_{n}$ módon jelöljük, ahol $|W| = \sum_{i = 0}^{n}|V_{i}|$.
    \end{definition*}

    \begin{definition*}
        Tegyük fel, hogy adott egy $n + 1$ halmazból álló véges $W=W_{0}, W_{1}, \ldots, W_{n}$ halmazsorozat. Ekkor a $W$ halmazsorozat fordítottját $\backwardhat W$ módon jelöljük, ahol $\backwardhat W = W_{n}, W_{n - 1}, \ldots, W_{0}$.
    \end{definition*}

    \begin{definition*}
        Tegyük fel, hogy adott két véges halmazsorozat, $M$ és $W$, ahol $|M| \leq |W|$. Azt mondjuk, hogy $W$ tartalmazza $M$-et, ha $M$ összefüggő részsorozata $W$-nek, azaz létezik olyan $0 \leq i < |W|$, hogy $W_{j} = M_{j}$ ($i \leq j < i + |M|$). Ezt $M \subseteq W$ módon jelöljük.
    \end{definition*}

    \begin{definition*}
        Tegyük fel, hogy adott egy $n + 1$ halmazból álló véges $W=W_{0}, W_{1}, \ldots, W_{n}$ halmazsorozat, valamint egy $D$ véges halmaz. Jelölje ekkor $W \,\setminus\, D$ azt a véges halmazsorozatot, melyet úgy állítunk elő, hogy $W$ minden elemének vesszük a különbségét $D$-vel, azaz $W \,\setminus\, D = W_{0} \,\setminus\, D, W_{1} \,\setminus\, D, \ldots, W_{n} \,\setminus\, D$.
    \end{definition*}

    \begin{definition*}
        Legyen $\mathscr{A}=(S, A)$ egy reverzibilis \textit{reaction system}. Azt mondjuk, hogy a $\mathscr{B}$ \textit{reaction system} interaktívan szimulálja $\mathscr{A}$ lépéseit, amennyiben a következő feltételek teljesülnek:
        \begin{itemize}
            \item
            Legyen $V$ egy olyan véges halmazsorozat, melyre teljesül, hogy létezik olyan $\pi$ \textit{interactive process} $\mathscr{A}$-ban, hogy $V \subseteq \textit{sts}(\pi)$. Ekkor létezik olyan $\sigma$ és $\tau$ \textit{interactive process} $\mathscr{B}$-ben, hogy $V \subseteq \textit{sts}(\sigma) \setminus \{ \, \rho \, \}$ és $\backwardhat V \subseteq \textit{sts}(\tau) \setminus \{ \, \rho \, \}$.
            \item
            Legyen $\sigma=(\gamma, \delta)$ egy tetszőleges \textit{interactive process} $\mathscr{B}$-ben a $\gamma = C_{0}, C_{1}, \ldots C_{n}$ bemeneti halmazokkal és az $\textit{sts}(\sigma)$ állapothalmazokkal. Ekkor $\textit{sts}(\sigma)$ felírható a $V_{0}, V_{1}, \ldots, V_{k}$ ($k \leq n$) véges halmazsorozatok felhasználásával, $\textit{sts}(\sigma)=V_{0}, \ldots, V_{k}$ formában, ahol az egyes $V_{i}$ halmazsorozatokra a következők egyike teljesül:
            \begin{itemize}
                \item
                Ha $i = 2m$ ($m \in \mathbb{N}$), akkor létezik olyan $\pi$ \textit{interactive process} $\mathscr{A}$-ban, hogy $V_{i} \,\setminus\, \{ \, \rho \, \}\subseteq \textit{sts}(\pi)$. Továbbá, ha $i > 0$ és $V_{i} = W_{j}, W_{j + 1}, \ldots, W_{j + k}$, akkor $C_{j} = \emptyset$.
                \item
                Ha $i = 2m + 1$ ($m \in \mathbb{N}$), akkor létezik olyan $\pi$ \textit{interactive process} $\mathscr{A}$-ban, hogy $\backwardhat V_{i} \,\setminus\, \{ \, \rho \, \} \subseteq \textit{sts}(\pi)$. Továbbá, ha $V_{i} = W_{j}, W_{j + 1}, \ldots, W_{j + k}$, akkor a $W_{p}$ ($p = j, j + 1, \ldots, j + k$) halmazok mindegyikére teljesül, hogy $W_{p} \setminus \{ \, \rho \, \}$ szerepel valamely megelőző $V_{q}$ előre szimulációt tartalmazó halmazsorozatban (azaz $q < i$ és $q = 2r$, $r \in \mathbb{N}$).
            \end{itemize}
            $\rho$ itt egy olyan segédszimbólum, mely egy szimulált hátrafelé lépést vált ki.
        \end{itemize}
        
    \end{definition*}

    \begin{remark*}
        Az előző definícióban a páratlan indexű $V_{i}$ halmazsorozatok jelölték a szimulált, visszafelé irányba tett lépéseket. Mivel $\mathscr{A}$ reverzibilis, ezért minden állapothalmaz csak egyféleképpen állítható elő. Ennél fogva, $V_{i}$ akárhány elemből is áll, azok csak olyan, egymást követő állapothalmazok lehetnek, melyek korábban megjelentek szimulált előre irányba tett lépésként.
    \end{remark*}

    \begin{remark*}
        Szimulált reverzibilitás esetén fennálhat az a probléma, hogy a szimuláló rendszer az eredeti rendszer kezdőállapotát megelőző állapotokba lép. A fenti definíció ezt nem teszi lehetővé, hiszen a szimulált visszafelé irányba tett lépések csak az eredeti rendszerben is szereplő lépéssorozatok fordítottjai lehetnek, mely kizárja a helytelen, a kezdőállapot elé tett lépéseket.
    \end{remark*}

    \begin{definition*}
        Legyen $A$ reakciók egy véges halmaza. Vezessük be ekkor a következő halmazokat:
        \begin{align*}
            R_{A} &= \bigcup\limits_{a \,\in\, A} R_{a} \\
            P_{A} &= \bigcup\limits_{a \,\in\, A} P_{a}
        \end{align*}
    \end{definition*}

    \begin{definition*}
        Legyen $A$ reakciók egy véges halmaza, valamint $B \subseteq A$. Vezessük be ekkor a következő halmazt:
        \begin{align*}
            C_{B} = \bigcup\limits_{\substack{A_{i} \subseteq A \\ P_{B} \subseteq P_{A_{i}}}} P_{A_{i}} \,\setminus\, P_{B}
        \end{align*}
    \end{definition*}

    \begin{theorem*}
        Legyen $\mathscr{A}$ egy reverzibilis \textit{reaction system}. Ekkor konstruálható olyan \textit{reaction system} mely interaktívan szimulálja $\mathscr{A}$ lépéseit.
    \end{theorem*}

    \begin{proof}
        A bizonyítás menete a következő. Először megkonstruálunk egy új, $\mathscr{B}$ \textit{reaction systemet} $\mathscr{A}$ alapján. Ezután belátjuk, hogy tetszőleges $\mathscr{B}$-beli $\sigma$ \textit{interactive processre} teljesül, hogy $\textit{sts}(\sigma)=V_{0}$ vagy $\textit{sts}(\sigma)=V_{0}V_{1}$, ahol $V_{0}$ egy olyan véges halmazsorozat, melyhez felírható egy $\mathscr{A}$-beli $\pi$ \textit{interactive process} úgy, hogy $V_{0} \setminus \mathcal{R} \subseteq \textit{sts}(\pi)$. Továbblépve, belátjuk, hogy ha $\sigma$-ra teljesül hogy $\textit{sts}(\sigma)=V_{0}V_{1}$, ahol $V_{0}$ megfelel az előző feltételnek, akkor $V_{1}$ felírható $V_{1}=V_{2}$ vagy $V_{1}=V_{2}V_{3}$ alakban, ahol $V_{2}$ egy olyan véges halmazsorozat, melyhez létezik egy $\mathscr{A}$-beli $\pi$ \textit{interactive process} úgy, hogy $\backwardhat V_{2} \setminus \mathcal{R} \subseteq \textit{sts}(\pi)$. Végül megmutatjuk, hogy $V_{3}$ további felbontásaira már analóg módon bizonyítható, hogy szerepelnek valamilyen $\mathscr{A}$-beli \textit{interactive process} állapothalmaz-sorozatában.

        Tegyük fel, hogy adott az $\mathscr{A} = (S, A)$ ($S = \Sigma_{p} \cup \Sigma_{c}$) reverzibilis \textit{reaction system}.
        
        Konstruáljuk meg ekkor a $\mathscr{B} = (T, B)$ \textit{reaction systemet} a következőképpen. Legyen $T = \Sigma_{p} \cup \Sigma^{\prime}_{c}$, ahol $\Sigma^{\prime}_{c} = \Sigma_{c} \cup \{ \, \rho \, \}$. A reakciók $B$ halmaza legyen $B = \forwardhat B \cup \backwardhat B$, ahol
        \begin{align*}
            \forwardhat B &= \{ \, \reaction{R_{a}}{I_{a} \cup \{ \, \rho \, \}}{P_{a}} \;|\; a \in A \, \},
        \end{align*}
        valamint
        \begin{align*}
            \backwardhat B = \{ \reaction{P_{A_{i}} \cup \{ \rho \}}{C_{A_{i}}}{R_{A_{i}}} \;|\; A_{i} \subseteq \textit{EN}_{A}(S) \}.
        \end{align*}
        Végül tegyük a következő megszorításokat a $\mathscr{B}$-beli \textit{interactive processek} bemeneti halmazaira vonatkozóan. Tetszőleges $\pi$ \textit{interactive process} esetén bármely $C_{i} \subseteq \Sigma^{\prime}_{c}$ bemeneti halmazra a következők egyike teljesül:
        \begin{itemize}
            \item
            ha $i = 0$, akkor $C_{0} \subseteq \Sigma_{c}$,
            \item
            ha $C_{i - 1} = \{\, \rho \,\}$, akkor $C_{i} = \emptyset$,
            \item    
            $C_{i} = \{\, \rho \,\}$,
            \item
            $C_{i} \subseteq \Sigma_{c}$.            
        \end{itemize}

        Most pedig lássuk be, hogy $\mathscr{B}$ egy olyan \textit{reaction system}, mely a megelőző definíciónak megfelelően, interaktívan szimulálja $\mathscr{A}$ lépéseit.

        Legyen $\sigma = (\gamma, \delta)$ egy tetszőleges \textit{interactive process} $\mathscr{B}$-ben a $\gamma = C_{0}, C_{1}, \ldots, C_{n}$ ($n \in \mathbb{N}$) bemeneti halmazokkal, $\delta = D_{1}, D_{2} \ldots D_{n}$ eredményhalmazokkal és az $\textit{sts}(\pi)=W_{0},W_{1},\ldots W_{n}$ állapothalmazokkal. Ekkor biztos, hogy valamely $0 \leq m_{0} \leq n$ értékre, $\rho \notin C_{i}$ ($i = 0, 1, \ldots, m_{0}$). Teljes indukcióval bizonyítsuk, hogy ekkor létezik olyan $\pi$ \textit{interactive process} $\mathscr{A}$-ban, hogy $\textit{sts}(\pi) = W_{0}, W_{1}, \ldots, W_{m}$.
        
        Mivel a $C_{0}$ bemeneti halmaz mindig olyan, hogy $C_{0} \subseteq \Sigma_{c}$, ezért $\sigma$ $W_{0}$ kezdőállapota biztosan szerepel valamilyen $\pi$ $\mathscr{A}$-beli \textit{interactive process} kezdőállapotaként. Ebből következik, hogy $m_{0} = 0$-ra a megelőző állítás biztosan igaz. Tegyük fel, hogy $m_{0} = k$-ra igaz az állítás, és ezt felhasználva lássuk be az $m_{0} = k + 1$ esetet! Ekkor tudjuk, hogy létezik olyan $\pi$ \textit{interactive proces} $\mathscr{A}$-ban, hogy $\textit{sts}(\pi) = W_{0}, W_{1}, \ldots, W_{k}$. Ez csak akkor lehetséges, ha $\rho \notin W_{k}$. Ekkor viszont $W_{k} \subseteq S$. Tekintsük most az erre a halmazra alkalmazható reakciókat! Mivel $\rho \notin W_{k}$, ezért csak $\forwardhat B$-beli reakciók lehetnek alkalmazhatók. Vegyük észre, hogy $\forwardhat B$ pontosan ugyanazokat a reakciókat tartalmazza, mint $A$, azzal a különbséggel, hogy az egyes \textit{inhibitor} halmazok ki lettek bővítve a $\rho$ elemmel. Mivel $\rho \notin W_{k}$, ezért ebben az esetben ezek az elemek irrelevánsak lesznek, amiből következik, hogy $\res_{\mathscr{A}}(W_{k}) = \res_{\mathscr{B}}(W_{k}) = D_{k + 1}$. Tehát azt már biztosan állíthatjuk, hogy létezik olyan $\pi = (\gamma_{\pi}, \delta_{\pi})$ \textit{interactive proces} $\mathscr{A}$-ban, hogy $\delta_{\pi} = D_{1}, D_{2}, \ldots D_{k}, D_{k + 1}$. Lássuk most a bemenetet! Tudjuk, hogy $\rho \notin C_{k + 1}$, azaz $C_{k + 1} \subseteq \Sigma_{c}$. Ekkor viszont $W_{k + 1} = C_{k + 1} \cup D_{k + 1} \subseteq S$, azaz létezik olyan $\pi$ \textit{interactive process} $\mathscr{A}$-ban, hogy $\textit{sts}(\pi)=W_{0}, W_{1}, \ldots, W_{k + 1}$. Vagyis az állítás teljesül $m_{0} = k + 1$-re is.  Ezzel egyúttal azt is bizonyítottuk, hogy a tetszőlegesen megválasztott $\sigma$ \textit{interactive process} állapothalmaz-sorozata mindig egy olyan $V_{0}$ halmazsorozatot tartalmaz prefixként, melyre igaz, hogy $V_{0} \subseteq \textit{sts}(\pi)$, valamilyen $\pi$ $\mathscr{A}$-beli \textit{interactive processre}.
        
        Lépjünk tovább, és tegyük fel, hogy $m_{0} < n$. Ekkor biztos, hogy valamely $m_{0} < m_{1} \leq n$ értékre $C_{i} = \{\, \rho \,\}$ ($i = m_{0} + 1, m_{0} + 2, \ldots, m_{1}$). Ismét a teljes indukció elvét használva bizonyítsuk, hogy ekkor létezik olyan $\pi$ \textit{interactive process} $\mathscr{A}$-ban, hogy a $V_{1} = W_{m_{0} + 1}, W_{m_{0} + 2}, \ldots, W_{m_{1}}$ halmazsorozatra teljesül, hogy $\backwardhat V_{1} \setminus \{\, \rho \,\} \subseteq \textit{sts}(\pi)$.

        Először tekintsük az indukció első lépését, azaz bizonyítsuk a fenti állítást $m_{1} = m_{0} + 1$-re. Az előző bizonyításból adódik, hogy ekkor biztosan létezik olyan $\pi = (\gamma_{\pi}, \delta_{\pi})$ \textit{interactive proces} $\mathscr{A}$-ban, hogy $\delta_{\pi} = D_{1}, D_{2}, \ldots D_{m_{0}}, D_{m_{1}}$. Mivel $C_{m_{1}} = \{ \, \rho \, \}$, ezért adódik, hogy $W_{m_{1}} \setminus \{ \,  \rho \, \} = (C_{m_{1}} \cup D_{m_{1}}) \setminus \{ \,  \rho \, \} = D_{m_{1}}$. Ez azt jelenti, hogy $W_{m_{1}} \setminus \{ \,  \rho \, \} = \res_{\mathscr{A}}(W_{m_{0}})$, amiből következik, hogy létezik olyan $\pi$ \textit{interactive process} $\mathscr{A}$-ban, hogy $\textit{sts}(\pi)=W_{0}, W_{1}, \ldots, W_{m_{1}}$. Ez egyúttal azt is jelenti, hogy a $V_{1} = W_{m_{1}}$ véges halmasorozatra teljesül, hogy $\backwardhat V_{1} \setminus \{ \,  \rho \, \} \subseteq \textit{sts}(\pi)$. Most tegyük fel, hogy $m_{1} = m_{0} + k$-ra igaz az állítás, és ezt felhasználva lássuk be az $m_{1} = m_{0} + k + 1$ esetet! Ekkor tudjuk, hogy a $V = W_{m_{0} + 1}, W_{m_{0} + 2}, \ldots W_{m_{0} + k}$ halmazsorozatra teljesül, hogy $\backwardhat V \setminus \{ \, \rho \, \} \subseteq \textit{sts}(\pi)$ valamilyen $\pi$ \textit{interactive processre} $\mathscr{A}$-ban, továbbá, hogy $C_{m_{0} + k} = \{ \, \rho \, \}$ és $C_{m_{0} + k + 1} = \{ \, \rho \,\}$. Nézzük meg, hogy milyen reakciók alkalmazhatók a $W_{m_{0} + k}$ halmazra! Mivel $\rho \in W_{m_{0} + k}$, ezért biztosan csak $\backwardhat B$-beli reakciókat kell tekintenünk. Mivel $\mathscr{A}$ reverzibilis, ezért egyértelmű, hogy egy $D$ eredményhalmaz mely $A_{i} \subseteq A$ reakcióhalmaz alkalmazásával állt elő. Ez azt jelenti, hogy $D$ alapján megmondható, mely reakciók kerültek alkalmazásra, amiből visszaállítható az a $W$ halmaz, melyre teljesül, hogy $\res_{\mathscr{A}}(W)=D$. Csupán össze kell fognunk az $A_{i}$-beli reakciókat, és elkészíteni egy olyan új reakciót, melynek reagens halmaza $D = P_{A_{i}}$, produktum halmaza pedig $W = R_{A_{i}}$. Vigyáznunk kell azonban arra, hogy ha létezik egy vagy több olyan $A_{j} \subseteq A$ halmaz, hogy $P_{A_{i}} \subset P_{A_{j}}$, akkor az összes $P_{A_{j}} \setminus P_{A_{i}}$-beli elem bekerüljön az új reakció \textit{inhibitor} halmazába. Vegyük észre, hogy $\backwardhat B$ pontosan ilyen módszerrel képzett reakciókat tartalmaz. Most már csak meg kell keresnünk azt a reakciót, mely alkalmazható $W_{m_{0} + k}$-ra. Visszanyúlva az indukciós feltételhez, tudjuk, hogy a $V = W_{m_{0} + 1}, W_{m_{0} + 2}, \ldots W_{m_{0} + k}$ halmazsorozatra teljesül, hogy $\backwardhat V \setminus \{ \, \rho \, \} \subseteq \textit{sts}(\pi)$ valamilyen $\pi$ \textit{interactive processre} $\mathscr{A}$-ban, ezért $W_{m_{0} + k} \setminus \{ \,  \rho \, \}$-ra két eset állhat fenn: vagy $\pi$ kezdőállapota, vagy egy $W$ állapothalmazból állt elő, valamilyen reakciók alkalmazásával. Előbbi esetben $W_{m_{0} + k} \setminus \{ \,  \rho \, \} \subseteq \Sigma_{c}$, melynek természetesen nincsen megelőző állapota. Utóbbi eset viszont azt jelenti, hogy $\res_{\mathscr{A}}(W) = W_{m_{0} + k} \setminus \{ \,  \rho \, \}$. Tehát azt a $b \in \backwardhat B$ reakciót kell alkalmazzuk, amelyre teljesül, hogy $P_{b} = W_{m_{0} + k} \setminus \{ \,  \rho \, \}$. Ilyen reakció mindig egyértelműen létezik: egyértelmű, hiszen $\mathscr{A}$ reverzibilis, azaz minden eredményhalmaz csak egyféleképpen állítható elő, és létezik, mert $V$ teljesíti az indukciós feltételt, tehát a rákövetkező eredményhalmaz valamilyen $\mathscr{A}$-beli $\pi$ \textit{interactive process} egy eredményhalmaza. Ezt a $b$ reakciót a $W_{m_{0} + k} \setminus \{ \,  \rho \, \} = P_{b}$ halmazva alkalmazva egy olyan $R_{b} = D_{m_{0} + k + 1}$ halmazt kapunk, hogy $\res_{\mathscr{A}}(D_{m_{0} + k + 1}) = W_{m_{0} + k} \setminus \{ \,  \rho \, \}$. Tehát $W_{m_{0} + k + 1} = R_{b} \cup \{ \, \rho \, \}$. Ez pedig pontosan azt jelenti, hogy ezek az állapothalmazok felírhatók egy olyan $V_{1} = W_{m_{0} + 1}, W_{m_{0} + 2}, \ldots, W_{m_{0} + k}, W_{m_{0} + k + 1}$ halmazsorozat formájában, melyre teljesül, hogy $\backwardhat V_{1} \setminus \{ \,  \rho \, \} \subseteq \textit{sts}(\pi)$ valamilyen $\pi$ \textit{interactive processre} $\mathscr{A}$-ban.

        Végül már csak azt kell bizonyítanunk, hogy ha $m_{1} < n$, akkor a fenti két esethez képest analóg módon bizonyítható, hogy megfelelő halmazsorozatok alkotják $\sigma$ fennmaradó részét. Ha $m_{1} < n$, akkor a $C_{m_{1} + 1}$ halmazról biztosan tudjuk, hogy $C_{m_{1} + 1} = \emptyset$

        Az előző levezetésből tudjuk, hogy $\res_{\mathscr{A}}(D_{m_{1} + 1}) = \res_{\mathscr{B}}(D_{m_{1} + 1}) = W_{m_{1}} \setminus \{ \,  \rho \, \}$. Mivel $C_{m_{1} + 1} = \emptyset$, ezért $\res_{\mathscr{B}}(W_{m_{1} + 1}) = W_{m_{1}} \setminus \{ \,  \rho \, \}$. Ez azt jelenti, hogy $W_{m_{1} + 1} \subseteq S$ biztosan olyan állapothalmaz, mely szerepel valamilyen $\mathscr{A}$-beli \textit{interactive process} állapothalmazainak sorozatában. De akkor, az előrelépésekre vonatkozó indukciót ismerve, tudjuk, hogy a következő eredményhalmaz is ilyen tulajdonságú lesz. Akkor pedig az azt követő halmazok is sorra teljesítik az előre- vagy hátralépésekre vonatkozó indukciót, ami által a tetszőlegesen megválasztott $\sigma$ \textit{interactive process} biztosan a szimulált reverzibilitás definíciójának megfelelő lesz.

        Ha pedig tetszőleges $\mathscr{B}$-beli \textit{interactive process} megfelel a definíciónak, akkor az azt jelenti, hogy $\mathscr{B}$ valóban interaktívan szimulálja $\mathscr{A}$ lépéseit.
    \end{proof}
\end{document}
