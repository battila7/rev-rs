\documentclass[12pt]{article}

\usepackage{t1enc}
\usepackage[utf8]{inputenc}
\usepackage[magyar]{babel}
\usepackage{amsmath}
\usepackage{amsfonts}
\usepackage{amsthm}
\usepackage{amssymb}
\usepackage{enumerate}
\usepackage[shortlabels]{enumitem}
\usepackage[autostyle]{csquotes}
\usepackage{graphicx}


\selectlanguage{magyar}

\DeclareQuoteAlias{dutch}{magyar}

\theoremstyle{definition}
\newtheorem{definition}{Definíció}
\newtheorem*{definition*}{Definíció}

\let\oldemptyset\emptyset
\let\emptyset\varnothing

\newcommand{\forwardhat}{\overset{\rightharpoonup}}
\newcommand{\backwardhat}{\overset{\leftharpoonup}}

\newcommand{\inc}{\textit{inc}}
\newcommand{\dec}{\textit{dec}}

\graphicspath{ {./images/} }


\title{Reverzibilis Reaction System}
\date{\today}

\begin{document}
    \maketitle

    \begin{definition*}
        Az $\mathcal{A} = (S, A)$ \textit{reaction system} reverzibilissé tehető, amennyiben teljesülnek a következő feltételek:
        \begin{enumerate}[label={(\arabic*)}]
            \item
            $\mathcal{A}$ nem tartalmaz olyan reakciókat, melyeknek jobb oldala átfedő: tetszőleges $i, j \in A$ reakció esetén $P_{i} \cap P{j} = \emptyset$, ha $i \neq j$.

            \item
            A kontextusból kapott szimbólumok nem állhatnak elő egy reakció produktumaként sem: ha $\pi = (\gamma, \delta)$ egy \textit{interactive process}, ahol $\gamma = C_{0}, C_{1}, \ldots, C_{n}$, $n \geq 1$, akkor bármely $C_{i}$ kontextus és $a \in A$ reakció esetén $C_{i} \cap P_{a} = \emptyset$.

            \item
            Az állapotok minden eleme részt vesz valamilyen reakcióban: ha $\pi$ egy \textit{interactive process}, ahol $\textit{sts}(\pi) = W_{0}, W_{1}, \ldots, W_{n}$, $n \geq 1$, akkor $\bigcup_{a \in \textit{en}(W_{i})} R_{a} = W_{i}$, $i \leq n$.
        \end{enumerate}
        Ekkor az $\mathcal{A}$-nak megfelelő reverzibilis \textit{reaction system} $\mathcal{A}_{\textit{rev}} = (S_{\textit{rev}}, A_{\textit{rev}})$, ahol
        \begin{align*}
            S_{\textit{rev}} &= S \cup \{ \, \rho \, \}, \\
            A_{\textit{rev}} &= \forwardhat A \cup \backwardhat A, \\ 
            \forwardhat A &= \{ \, (R_{a}, I_{a} \cup \{ \, \rho \, \}, P_{a}) \; : \; a \in A \, \}, \\
            \backwardhat A &= \{ \, (P_{a} \cup \{ \, \rho \, \}, \emptyset, R_{a}) \; : \; a \in A \, \}.
        \end{align*}
        $\rho$ egy speciális szimbólum (tehát $\rho \notin P_{a}, a \in A_{\textit{rev}}$), mely egy visszafelé irányba tett számítási lépésre kényszeríti a rendszert.
    \end{definition*}
\end{document}
