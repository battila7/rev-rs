\documentclass[12pt]{article}



\usepackage{t1enc}
\usepackage[utf8]{inputenc}
\usepackage[magyar]{babel}
\usepackage{amsmath}
\usepackage{amsfonts}
\usepackage{amsthm}
\usepackage{amssymb}
\usepackage{enumerate}
\usepackage{mathtools}
\usepackage[shortlabels]{enumitem}
\usepackage[autostyle]{csquotes}
\usepackage{graphicx}
\usepackage{mathrsfs}



\selectlanguage{magyar}

\DeclareQuoteAlias{dutch}{magyar}

\theoremstyle{definition}
\newtheorem{definition}{Definíció}
\newtheorem*{definition*}{Definíció}

\theoremstyle{remark}
\newtheorem{remark}{Megjegyzés}
\newtheorem*{remark*}{Megjegyzés}

\theoremstyle{plain}
\newtheorem{theorem}{Tétel}
\newtheorem*{theorem*}{Tétel}

\let\oldemptyset\emptyset
\let\emptyset\varnothing

\newcommand{\forwardhat}{\overset{\rightharpoonup}}
\newcommand{\backwardhat}{\overset{\leftharpoonup}}

\newcommand{\inc}{\textit{inc}}
\newcommand{\dec}{\textit{dec}}

\newcommand{\en}{\; \textit{en} \;}
\newcommand{\excl}{\; \textit{excl} \;}
\newcommand{\incl}{\; \textit{incl} \;}

\newcommand{\reaction}[3]{
    (#1, \, #2, \, #3)
}

\graphicspath{ {./images/} }



\title{Reverzibilis Reaction System}
\date{\today}

\begin{document}
    \maketitle

    \section*{Előzetes ismeretek}

    \begin{definition*}
        Legyen $\mathscr{A} = (S, A)$ egy \textit{reaction system}. Egy $\mathscr{A}$-beli \textit{interactive process} véges sorozatok olyan $\pi = (\gamma, \delta)$ párja, amelyben $\gamma = C_{0}, C_{1}, \ldots C_{n}$, $\delta = D_{1}, \ldots D_{n}$, $n \geq 1$, ahol $C_{0}, \ldots C_{n}$, $D_{1}, \ldots D_{n} \subseteq S$, $D_{1} = \textit{res}_{\mathscr{A}}(C_{0})$ és $D_{i} = \textit{res}_{\mathscr{A}}(D_{i - 1} \cup C_{i - 1})$, ha $2 \leq i \leq n$.
    \end{definition*}

    \begin{definition*}
        Legyen $a = \reaction{R_{a}}{I_{a}}{P_{a}}$ egy reakció, $T$ pedig egy véges halmaz. Azt mondjuk, hogy $a$ alkalmazható $T$-re, ha $R_{a} \subseteq T$ és $I_{a} \cap T = \emptyset$. Erre a továbbiakban az $a \en T$ jelölést használjuk.
    \end{definition*}
    
    \begin{remark*}
        A következőkben, az egyszerűbb olvashatóság érdekében, feltesszük, hogy ha adott egy $a \in A$ reakció, akkor annak komponenseit $a$ alsó indexbe helyezésével helyezésével jelöljük, az $a = \reaction{R_{a}}{I_{a}}{P_{a}} \in A$ kifejezés kiírása nélkül.
    \end{remark*}

    \begin{definition*}
        Legyen $a$ és $b$ ($a \neq b$) két reakció. Azt mondjuk, hogy $a$ és $b$ nem alkalmazhatók együtt, ha nem létezik olyan $T$ véges halmaz, melyre egyidejűleg $a \en T$ és $b \en T$ teljesülne. Ekkor $R_{a} \cap I_{b} \neq \emptyset$ vagy $R_{b} \cap I_{a} \neq \emptyset$. Erre a kapcsolatra a továbbiakban az $a \excl b$ jelölést alkalmazzuk.
    \end{definition*}

    \begin{definition*}
        Legyen $a$ és $b$ ($a \neq b$) két reakció. Azt mondjuk, hogy $b$ $a$-val együtt mindig alkalmazható, ha bármely $T$ véges halmazra, melyre $a \en T$ teljesül, $b \en T$ is teljesül. Ekkor $R_{b} \subseteq R_{a}$ és $I_{b} \subseteq I_{a}$. A továbbiakban ezt a kapcsolatot $a \incl b$ módon jelöljük.
    \end{definition*}

    \section*{Reverzibilis Reaction System}

    \begin{definition*}
        Legyen $\mathscr{A} = (S, A)$ egy \textit{reaction system} és $\pi$ egy olyan \textit{interactive process} $\mathscr{A}$-ban, amelynek állapotait $\textit{sts}(\pi)=W_{0},W_{1},\ldots W_{n}$ módon jelöljük. $\pi$ reverzibilis, amennyiben minden $W_{i}$ ($1 \leq i \leq n$) állapotára teljesül, hogy $\nexists \; W \in \mathcal{P}(S) : res_{\mathscr{A}}(W) = W_{i} \; \wedge W \neq W_{i - 1}$.
    \end{definition*}

    \begin{definition*}
        Egy $\mathscr{A}$ \textit{reaction system} reverzibilis, amennyiben minden $\mathscr{A}$-beli $\pi$ \textit{interactive process} reverzibilis.
    \end{definition*}

    \begin{remark*}
        A továbbiakban, az általánosság elvesztése nélkül, kizárólag olyan $\mathscr{A} = (S, A)$ \textit{reaction systemeket} fogunk tekinteni, melyek nem tartalmaznak azonos feltételek mellett alkalmazható reakciókat. Ez azt jelenti, hogy nincs két olyan $a, b\in A$ ($a \neq b$) reakció, melyekre teljesülne, hogy $R_{a} = R_{b}$ és $I_{a} = I_{b}$.
    \end{remark*}

    \begin{theorem*}
        Az $\mathscr{A} = (S, A)$ \textit{reaction system} reverzibilis, amennyiben teljesülnek a következő feltételek:
        \begin{enumerate}[label={(\arabic*)}]
            \item
            Egyértelmű, hogy egy állapot mely reakciók alkalmazásával állt elő. Azaz, tetszőleges $a, b \in A$ ($a \neq b$) reakciópár esetén a következők egyike teljesül:
            \begin{itemize}
                \item
                $a$ és $b$ produktumai nem átfedők, azaz $P_{a} \cap P_{b} = \emptyset$.

                \item
                $a$ produktuma tartalmazza $b$ produktumát is, azonban $a$ és $b$ nem alkalmazhatók egyszerre, tehát $P_{b} \subset P_{a}$ és $a \excl b$.

                \item
                $a$ és $b$ produktumai megegyezők, azonban van olyan $c \in A$ reakció, mely $a$-val együtt mindig, $b$-vel együtt azonban sosem alkalmazható, és melynek produktuma tartalmaz egy vagy több, $b$ produktumában nem szerepelő elemet. Ekkor $a \incl c$, $c \excl b$ és $P_{b} \cup P_{c} \neq P_{b}$.
            \end{itemize} 

            \item
            A kontextusból kapott szimbólumok nem állhatnak elő egy reakció produktumaként sem: ha $\pi = (\gamma, \delta)$ egy \textit{interactive process}, ahol $\gamma = C_{0}, C_{1}, \ldots, C_{n}$, $n \geq 1$, akkor bármely $C_{i}$ kontextus és $a \in A$ reakció esetén $C_{i} \cap P_{a} = \emptyset$.

            \item
            Az állapotok minden eleme részt vesz valamilyen reakcióban: ha $\pi$ egy \textit{interactive process}, ahol $\textit{sts}(\pi) = W_{0}, W_{1}, \ldots, W_{n}$, $n \geq 1$, akkor $\bigcup_{a \in \textit{en}(W_{i})} R_{a} = W_{i}$, $i \leq n$.
        \end{enumerate}
    \end{theorem*}

    \subsection*{Példák}

    \subsubsection*{Reverzibilis bináris számláló}

    Reverzibilis \textit{reaction system} felhasználásával megvalósítható egy olyan bináris számláló, melynek értéke az előrefelé számítás során növekszik, míg a hátrafelé számítás során csökken.

    Először is tegyük fel, hogy adott egy $n > 0$ egész. $n$ jelenti a számláló bithosszát. Ekkor a \textit{reaction system} alaphalmaza a következő lesz:
    \begin{align*}
        S_{n} = \{ p_{0}, p_{1}, \ldots, p_{n - 1}\} \cup \{ \textit{inc} \}.
    \end{align*}
    A fenti halmaz $p_{i}$ elemei reprezentálják az egyes bitek beállított (azaz $1$ értékű) állapotát, míg az $\textit{inc}$ elemmel a számláló értékének növelését válthatjuk ki.

    A számábrázolás tehát a következőképpen alakul. Tegyük fel, hogy a \textit{reaction system} egy $M \subseteq S$ állapotban van. Ekkor, ha $p_{i} \in M$, akkor az $i$-edik pozíción levő bit $1$ értékkel, amennyiben pedig $p_{i} \notin M$, akkor $0$ értékkel rendelkezik. Például, ha $n = 4$ és $M = \{p_{2}, p_{0}\}$, akkor a \textit{reaction system} állapota a $0101$ bináris számot írja le.
    
    Előrefelé számítás során az $\textit{inc}$ elemet használhatjuk a számláló értékének eggyel történő növelésére. Egyszerű példát tekintve, ez azt jelenti, hogy amennyiben a \textit{reaction system} egy $\{p_{1}, p_{0}, \textit{inc}\}$ állapotban van, akkor valamely reakciók végrehajtása után a $\{ p_{2} \}$ állapotba kell kerülnie.

    Folytassuk tehát az említett működéshez szükséges reakciók megadásával! Legyen $n > 0$ adott, ekkor a reakciók $A_{n}$ halmaza a következő elemekből áll:
    \begin{align*}
        a_{i} = \reaction{\{ \inc \} \cup O_{i}}{Z_{i}}{O_{i + 1}}, \qquad 0 \leq i < 2^{n} - 2,
    \end{align*}
    ahol
    \begin{align*}
        O_{i} &= \{ \; p_{j} \; : \; \text{a $j$-edik bit értéke $1$ $i$ bináris felbontásában} \; \}, \\
        Z_{i} &= S \setminus\{ \inc \} \setminus O_{i}.
    \end{align*}
    Az egyes reakciók megfelelnek a számláló értékének $i$-ről $i + 1$-re történő növelésének.

    Az $n$-bites számlálónak megfelelő \textit{reaction system} ekkor $\mathcal{B}_{n}=(S_{n}, A_{n})$.

    Tekintsünk most egy példát! Tegyük fel, hogy egy kétbites számlálót szeretnénk készíteni, azaz $n = 2$. Az $S_{2}$ alaphalmaz ekkor a $\{p_{1}, p_{0}, \inc\}$ elemekből áll, a reakciók $A_{2}$ halmazát pedig az
    \begin{align*}
        a_{0} &= \reaction{\{ \inc \}}{\{ p_{1}, p_{0} \}}{\{ p_{0} \}}, \\
        a_{1} &= \reaction{\{ \inc, p_{0} \}}{\{ p_{1} \}}{\{ p_{1} \}}, \\
        a_{2} &= \reaction{\{ \inc, p_{1} \}}{\{ p_{0} \}}{\{ p_{1}, p_{0} \}},
    \end{align*}
    elemek alkotják.

    Ha az egymást követő kontextushalmazok sorra az \inc növelő elemből állnak, akkor a $\mathcal{B}_{2} = (S_{2}, A_{2})$ \textit{reaction system} a következő állapotokat fogja kiszámolni:
    \begin{align*}
        \emptyset \rightarrow \{p_{0}\} \rightarrow \{p_{1}\} \rightarrow \{p_{1}, p_{0}\}.
    \end{align*}

    \section*{A reverzibilitás szimulálása}

    A következőkben egy olyan \textit{reaction systemet} írunk le, mely interaktív előrefelé számításokkal képes szimulálni egy, a fenti tétel szerinti reverzibilis \textit{reaction system} előre- és hátrafelé irányba tett lépéseit.

    Ez egy olyan számítási szemantikát jelenít meg, ahol tetszés szerint lehetőségünk van a megelőző lépéseink visszavonására, majd új számítási utak kiválasztására.

    \begin{theorem*}
        Legyen $\mathscr{A} = (S, A)$ egy olyan \textit{reaction system}, mely a megelőző tétel szerint reverzibilis. Ekkor létezik olyan $\mathscr{R} = (T, B)$ \textit{reaction system}, mely $\mathscr{A}$ reverzibilitását szimulálja, mégpedig a következőképpen.
        
        A $T$ alaphalmaz az $\mathscr{A}$ \textit{reaction system} alaphalmaza, kiegészülve egy speciális $\rho$ szimbólummal, mely egy szimulált visszafelé lépés kiváltására szolgál:
        \begin{align*}
            T = S \cup \{ \rho \}.
        \end{align*}

        A $B$ reakcióhalmaz a következő módon adott:
        \begin{align*}
            B &= \forwardhat B \cup \backwardhat B, \\
            \forwardhat B &= \{ (R_{a}, I_{a} \cup \{ \rho \}, P_{a}) : a \in A\}, \\
            \backwardhat B &= \{ \textit{rev}(a) : a \in A \},
        \end{align*}
        ahol a $\textit{rev}$ függvény az alábbiak szerint állít elő egy új reakciót:
        \begin{enumerate}[label={(\arabic*)}]
            \item
            Ha $a \in A$ olyan, hogy nincs olyan $b \in A$ $(a \neq b)$, hogy $P_{a} \cap P_{b} \neq \emptyset$, akkor
            \begin{align*}
                \textit{rev}(a) = \reaction{P_{a} \cup \{ \rho \}}{\emptyset}{R_{a}}.
            \end{align*}

            \item
            Ha $a \in A$ olyan, hogy vannak olyan $b_{i} \in A$ $(a \neq b_{i})$ reakciók, melyek mindegyikére teljesül, hogy $P_{a} \cap P_{b_{i}} \neq \emptyset$ és $P_{a} \neq P_{b_{i}}$, akkor
            \begin{align*}
                \textit{rev}(a) = \reaction{P_{a} \cup \{ \rho \}}{\bigcup\limits_{i}P_{b_{i}} \setminus P_{a}}{R_{a}}.
            \end{align*}

            \item
            Ha az $a_{i} \in A$ $(0 \leq i < n$, ahol $n > 0$ egész$)$ reakciók olyanok, hogy $P_{a_{0}} = P_{a_{1}} = \cdots = P_{n - 1}$, és legfeljebb egy kivételével mindegyik $a_{i}$ reakcióhoz létezik olyan $b_{i}$ reakció, melyre teljesül, hogy
            \begin{align*}
                a_{i} &\incl b_{i}, \\
                b_{i} \excl a_{j} \quad & \quad 0 \leq j < n, i \neq j,
            \end{align*}
            akkor
            \begin{align*}
                rev(a_{i}) = \reaction{P_{a_{i}} \cup P_{b_{i}} \cup \{ \rho \}}{\bigcup\limits_{\substack{ \mathllap{0 \, \leq} \, j \, \mathrlap{< \, n} \\ \mathllap{i} \, \neq \, \mathrlap{j} }}P_{b_{j}}}{R_{a_{i}}}.
            \end{align*}
        \end{enumerate}
    \end{theorem*}
\end{document}

